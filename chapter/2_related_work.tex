\chapter{関連研究}
\label{chap:related_work}


\section{LaTeXについて}
\label{sec:latex}

\LaTeX にも色々あるらしいです.以前のテンプレートはpLaTeXを使っていましたが,今回はLuaLaTeXを使っています.
LuaLaTeXを使うと日本語で書くことが出来ます.
%-------------------------------------------------------------------------
\section{お役立ち情報}
\subsection{数式の書き方}

\texttt{equation}環境を使うと数式を書くことができます.
\begin{equation}
  y = ax + b
\end{equation}
\texttt{\$}を使うとインライン数式$y=ax+b$を書くことができます.
複数行の数式は\texttt{split}を使う
ことができます.
\begin{equation}
  \begin{split}
    y & = a ( x + b ) \\
      & = ax + ab
  \end{split}
  \label{eq:equation1}
\end{equation}

\subsection{図の書き方}
\texttt{figure}環境を使うと\cref{fig:figure}のように図を書くことができます.図を自分で作る場合はpdfで作成すると良いらしいです.
\begin{figure}[h]
  \centering
  \includegraphics[width=0.8\linewidth]{fig/cat_kotatsu_neko.pdf}
  \caption{図にはPDF形式の画像を使ってください}
  \label{fig:figure}
\end{figure}

\subsection{表の書き方}
\texttt{table}環境を使うと\cref{tab:table}のように表を書くことができます.
\begin{table}[h]
  \centering
  \begin{tabular}{@{}cccc@{}}
    \toprule
    method        & 評価指標その1 $\downarrow$ & その2   $\uparrow$ \\
    \midrule
    先行研究          & $100$                & $2$              \\
    \textbf{ours} & $\mathbf{0.1}$       & $\mathbf{500}$   \\
    \bottomrule
  \end{tabular}
  \caption{良い性能のものを太字にするとよいです}
  \label{tab:table}
\end{table}


\subsection{参照の仕方}

以前のテンプレートでは\texttt{\textbackslash figref}や\texttt{\textbackslash tabref}を使っていました.引き続きこれらを使うこともできますが,\texttt{\textbackslash cref}を使って参照する対象に関係なく一括で\cref{fig:figure}や\cref{tab:table}のように参照することができるようにしました.
\begin{itemize}
  \item \cref{chap:related_work}
  \item \cref{sec:latex}
  \item \cref{eq:equation1}
  \item \cref{chap:suppl}
\end{itemize}

\subsection{参考文献の書き方}

参考文献は\texttt{\textbackslash cite}を使って引用することができます\cite{bib1}.
複数まとめて引用もできます\cite{bib1,bib2}.